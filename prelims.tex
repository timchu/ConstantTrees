\section{Preliminaries}
\subsection{Electrical Flows and Effective Resistance}

Let graph G = (V, E, $c$) have edge weights $c_e$, where $c_e$ is the conductance of each edge.  Define the resistance $r_e$ on each edge to be $\frac{1}{c_e}$. 

Let $L_G$ be the Laplacian of graph $G$. Let $\vv \in \mathbb{R}^|V|$ be the vector of voltages on the vertices of $V$.

Let the vector $\chiv$ denote the vector of excess demand on each vertex. It's well known in the theory of electrical networks that
\begin{equation}
L_G \vv = \chiv,
\end{equation}

or equivalently,
\begin{equation}
\vv = L_G^{+} \chiv
\end{equation}

where $L_G^{+}$ is the Moore-Penrose pseudo-inverse of $L_G$.

For edge $e = (i, j)$ with $i, j \in V$, the effective resistance $R_e(G)$ is defined as

\begin{equation}
R_e(G) = \frac{\chiv^TL_G^{+} \chiv}{2}
\end{equation}

where 
\begin{equation}\label{chiv}
   \chiv := \left\{
     \begin{array}{ll}
       1  &  x = i\\
       -1 &  x = j\\
       0  & \text{otherwise} 
     \end{array}
   \right.
\end{equation} 

The effective resistance of edge $e$ can be interpreted as the voltage drop across that edge given an flow of $1$ unit of current from $i$ to $j$. 

When the underlying choice of graph $G$ is clear, $R_e(G)$ will be shortened to $R_e$.

\begin{lemma} \label{effectresist}
For all graphs $H$ that spectrally sparsify $G$, 
\begin{equation}
\left(\frac{1}{1+\epsilon}\right) R_e(H) < R_e(G) < (1+\epsilon) R_e(H)
\end{equation}

\begin{proof} This follows immediately from Equation \ref{1}, and substituting
$$R_e(G) = \frac{\chiv^TL_G^{+} \chiv}{2}$$
and
$$R_e(H) = \frac{\chiv^TL_H^{+} \chiv}{2}.$$
\end{proof}
where $R_e(H)$ denotes the effective resistance of edge $e$ in $H$ and $R_e(G)$ denotes the effective resistance of $e$ in $G$.
\end{lemma}
\subsection{The Spielman-Srivastava Sparsifier}

Spielman and Srivastava showed in \cite{SpielmanS08} that any graph can be sparsified with high probability using the following routine, for a large enough constant $C$: 

\begin{itemize}
\item For each edge, assign it a probability $p_e := \frac{R_ec_e}{(n-1)}$, where $R_e$ is the effective resistance of that edge and $c_e$ is the conductance (the inverse of the actual resistance) of that edge. Create a distribution on edges where each edge occurs with probability equal to $p_e$.
\item Weight each edge to have conductance $\frac{c_e\epsilon^2}{(C n \log n) p_e}$, and sample $Cn \log n/\epsilon^2$ edges from this distribution.
\end{itemize}

We call such a scheme a Spielman-Srivastava sparsifying routine. Note that this scheme allows for multiple edges between any two vertices.
\begin{remark} Sampling by \textit{approximate} effective resistances (as Spielman and Srivastava did in their original paper \cite{SpielmanS08}) will work in place of using exact values for effective resistances. The results in our paper will still go through; an approximation will still ensure that every edge has a relatively large weighting, which is what the result in our paper depends on.
\end{remark}

\subsection{The Marcus-Spielman-Srivastava Sparsifier}
The following scheme from \cite{Srivastava13} produces a sparsifier with non-zero probability, for sufficiently large constants $C$:

\begin{itemize}
\item For each edge, assign it a probability $p_e := \frac{R_ec_e}{(n-1)}$, where $R_e$ is the effective resistance of that edge and $c_e$ is the conductance (inverse of actual resistance) of that edge. Create a distribution on edges where each edge occurs with probability equal to $p_e$.
\item Weight each edge to have conductance $\frac{c_e\epsilon^2}{p_e(Cn)}$, and sample $C n/\epsilon^2$ edges from this distribution.
\end{itemize}

We call such a scheme a Marcus-Spielman-Srivastava sparsifying routine. Note that this scheme allows for multiple edges between any two vertices. 

Note that the probability this routine returns a sparsifier may be expontentially small, and there is no known efficient algorithm to actually find such a sparsifier, making the \cite{SpielmanS08} result more algorithmically relevant.
\subsection{Uniform Sparsity and Low Arboricity}

\begin{definition}
The \textbf{arboricity} of a graph $G$ is the equal to the minimum number of forests its edges can be decomposed into.
\end{definition}
\begin{definition}
A graph G = (V, E, $c$) is said to be \textbf{$c$-uniformly sparse} if, for all subsets $V' \subset V$, the subgraph induced on $G$ by $V'$ contains no more than $c \cdot |V'|$ edges.
\end{definition}

\begin{lemma} \label{lowarb}
Uniform Sparsity implies Low Arboricity. That is, if $G$ is $c$-uniformly sparse, then the arboricity of $G$ is no greater than $2c$.
\end{lemma}
This statement is proven in \ref{lem_treelike}.

