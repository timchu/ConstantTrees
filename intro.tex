\section{Introduction}

\textbf{Graph Sparsifiers.} 
Let $G$ be a graph with vertices $V$, and
edges $E$, where $|V| = n$ and $|E| = m$. If $H$ is similar to $G$ under
some appropriate measure, then $H$ can be used as a proxy for $G$ in
computations. Benczur and Karger~\cite{BenczurK96} introduced the notion of a
graph sparsifier (called a cut-sparsifier) where $H$ is a cut-sparsifier
of $G$ (where $H$ has the same vertex set) if every cut in $H$ is within a $(1+\eps)$
factor of the corresponding cut in $G$. 
Spielman and Teng~\cite{SpielmanT04}
introduced the notion of graph sparsification in the spectral setting.
The Laplacian of $G$ is the unique symmetric matrix $L_G$ such that for
all vectors $x \in \mathbb{R}^n$, we have $x^T L_G x = \sum_{(u,v) \in
E} (x_u - x_v)^2$. $H$ is an $\eps$-spectral-sparsifier of $G$ if:

\begin{equation}\label{1}
  (1-\eps) \xx^TL_G\xx \leq \xx^TL_H\xx \leq (1+\epsilon)\xx^TL_G\xx.
\end{equation}

Spielman and Teng~\cite{SpielmanT04} showed that $\eps$-spectral-sparsifiers
can be constructed in
\\
$O(n \log^{O(1)} n / \eps^2)$ time. They also
demonstrated that a spectral sparsifier is a cut-sparsifier, by
restricting $\xx \in \{0, 1\}^n$.  Spielman and Srivastava later proved
that $\eps$-spectral sparsifiers with $O(n \log^2 n / \eps^2)$ edges can be
constructed in nearly linear time via leverage score sampling~\cite{SpielmanS08}.

\textbf{Arboricity.} For a graph $G$, the arboricity $A(G)$ of the graph
is the smallest integer $k$ such that there exists forests $T_1, T_2,
\ldots T_k$ which are subgraphs of $G$ such that their union is $G$.
This measure is equivalent, up to a factor of 2, to the maximum average
degree of any subgraph. Hence, arboricity can be viewed as a measure of
uniform sparsity.

Arboricity has been studied in the context of local
algorithms~\cite{Solomon18}, fixed-parameter
tractable algorithms~\cite{LinSS12, Solomon18}, minor-closed graph
families, and more. A considerable range of algorithms are known to run quickly
on low-arboricity graphs, including approximate vertex cover,
approximate dominating set, approximate independent set, approximate
maximum matching, and more ~\cite{Solomon18, Suomeloa12, KuhnW05, LinSS12, GoelG06}. Furthermore, low arboricity graphs imply a
vertex with low degree, which allows for fast Schur complementing (which
is analogous to vertex sparsification)~\cite{AbrahamDKKP16}.

% Low arboricity graphs have been used extensively in the last 40 years. 
% ~\tim{Insert 10000 citations.}.

\subsection{Contributions and Previous Work}

We show that any graph $G$ admits a subgraph $H$ that is a
$\eps$-spectral-sparsifiers with arboricity
$O(1/\eps^2)$. We also show that a range of spectral sparsifiers,
including
the Spielman Srivastava sparsifier ~\cite{SpielmanS08} have nearly
constant arboricity. This implies that all algorithms that run quickly on low
arboricity graphs run quickly on such sparsifiers.

Our paper's main lemma, with a simple proof,
is showing that for \textit{any} graph, the subset of edges
with leverage scores higher than $\alpha$ form a subgraph with
arboricity at most $O(1/ \alpha)$.

\begin{lemma} \label{lem:main} For any graph $G$, the subgraph $H$ of
  edges with leverage score $ > \alpha$ has arboricity $O(1 / \alpha)$.
\end{lemma}

This directly implies the following
theorem:

\begin{theorem}\label{thm:main}
  For any graph $G$ whose leverage scores are all $ > \alpha$, graph $G$
  has arboricity $O(1/ \alpha)$.
\end{theorem}

This theorem has the following immediately corollaries: 
\begin{corollary} \label{cor:ss}
  Any $\eps$-spectral-sparsifier constructed using the procedure of Spielman and
  Srivastava~\cite{SpielmanS08} has arboricity at most $O(\log n /
  \eps^2)$.
\end{corollary}

\begin{corollary}\label{cor:mss}
  There exists an
  $\eps$-spectral-sparsifier with $O(1/\eps^2)$ arboricity for any graph
  $G$.
\end{corollary}

\begin{corollary}\label{cor:kl}
  Any $\eps$-spectral-sparsifier constructed using the procedure of Kelner and
  Levin~\cite{KelnerL13} (rigorously proven in~\cite{Pachocki16, KyngPPS16}) has arboricity at most
  $O(\log n / \eps^2)$.
\end{corollary}
This is as tight a bound as we can hope to get for each class of
sparsifiers.
  
\begin{corollary}\label{lem:res-sparse}
  For any graph $G$, there exists a subgraph $H$ with arboricity $O(\log^{O(1)} n / \eps)$
  such that for all pairs of vertices $u$ and $v$ in $V(G)$, the effective resistance
  between them in graph $H$ is a $(1+\eps)$ approximation of the
  effective resistance between $u$ and $v$ in $G$.
\end{corollary}
Corollary~\ref{lem:res-sparse} reduces the $\epsilon$ dependency of the
arboricity, compared to the sparsifiers in
Corollaries~\ref{cor:ss},~\ref{cor:mss}~\ref{cor:kl}.

Proofs of our main theorem, and its corollaries, will be given in
Section~\ref{sec:main} \tim{Build this section}.

One other new result in our paper is that we show cut queries can be
calculated in nearly optimal time on low-arboricity graphs.
As shown by Andoni, Krauthgamer, and Woodruff in \cite{AndoniKW14},
any sketch of a graph that w.h.p.
preserves all cuts in an $n$-vertex graph must be of size
$\Omega(n/\epsilon^2)$ bits.
Using Spielman-Srivastava sparsifiers, we can achieve the
nearly optimal query time $O\left(|S|\frac{\log n}{\epsilon^2}\right)$
when estimating the boundary of $S \subseteq V$, compared to the trivial
query time of $O\left(n\frac{\log n}{\epsilon^2}\right)$. This proof
hinges on the low arboricity of Spielman Srivastava sparsifiers.
This is formalized in Theorem~\ref{thm:cut-query}.

Besides for cut queries, there is a long list of algorithms that run quickly on low arboricity
graphs. These 
which includes approximate maximum matching, approximate
dominating set, and approximate vertex cover~\cite{Solomon18,
Suomeloa12, KuhnW05, LinSS12, GoelG06}. Our result immediately implies
that fast algorithms for computing any of these, exist given any of the
spectral sparsifiers specified above. Approximate vertex cover on low-arboricity spectral
sparsifiers has been used by Abraham et.al. in~\cite{AbrahamDKKP16} to
compute $(1-\eps)$ undirected maximum bipartite flow.
It remains an open question what the output of these algorithms on a
spectral sparsifier $H$ of $G$, imply
about the original graph $G$.  

The existence of $O(\log^2 n / \eps^2)$ sparsifiers has been previously
implied by Kyng and Song~\cite{KyngS18}, which shows that a $O(\log^2
n / \eps^2)$ random spanning trees with edges re-weighted by their
effective conductance, form a spectral sparsifier with high
probability. It was previously known that this many random spanning
trees was a cut-sparsifier with high probability~\cite{FungH10}. 
