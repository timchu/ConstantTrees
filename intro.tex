\section{Introduction}

\textbf{Graph Sparsifiers.} 
Let $G$ be a graph with vertices $V$, and
edges $E$, where $|V| = n$ and $|E| = m$. If $H$ is similar to $G$ under
some appropriate measure, then $H$ can be used as a proxy for $G$ in
computations. Benczur and Karger~\cite{BK96} introduced the notion of a
graph sparsifier (called a cut-sparsifier) where $H$ is a cut-sparsifier
of $G$ (where $H$ has the same vertex set) if every cut in $H$ is within a $(1+\eps)$
factor of the corresponding cut in $G$. 
Spielman and Teng~\cite{SpielmanT04} \tim{Insert this citation}
introduced the notion of graph sparsification in the spectral setting.
The Laplacian of $G$ is the unique symmetric matrix $L_G$ such that for
all vectors $x \in \mathbb{R}^n$, we have $x^T L_G x = \sum_{(u,v) \in
E} (x_u - x_v)^2$. $H$ is an $\eps$-spectral-sparsifier of $G$ if:

\begin{equation}\label{1}
  (1-\eps) \xx^TL_G\xx \leq \xx^TL_H\xx \leq (1+\epsilon)\xx^TL_G\xx.
\end{equation}

Spielman and Teng \cite{SpielmanT04} showed that spectral sparsifiers
can be constructed in $O(n \log^{O(1)} n / \eps^2)$ time. They also
demonstrated that a spectral sparsifier is a cut-sparsifier, by
restricting $\xx \in \{0, 1\}^n$. 

\textbf{Arboricity.} For a graph $G$, the arboricity $A(G)$ of the graph
is the smallest integer $k$ such that there exists forests $T_1, T_2,
\ldots T_k$ which are subgraphs of $G$ such that their union is $G$.
Arboricity has been studied in the context of local algorithms
\tim{Citation needed}, real-world sparse graphs, fixed-parameter
tractable algorithms \tim{Citation needed}, minor-closed graph families,
and more. They have been applied in \tim{Cite Richard's papers}.
Furthermore, a considerable range of algorithms are known to run quickly
on low-arboricity graphs, including approximate vertex cover,
approximate dominating set, approximate independent set, approximate
maximum matching, and more.

In our paper, we show that a range of spectral sparsiifers whose
construction is based on leverage scores, have constant
or nearly constant
arboricity. The existence of $O(\log^2 n / \eps^2)$ sparsifiers is
implied by~\cite{KyngS18}, and low arboricity spectral sparsifiers are
constructed in \tim{Cite the Richard thing}. Our paper's core
contribution is showing that for \textit{any} graph, the set of edges
with leverage scores higher than $\alpha$ form a subgraph with
arboricity at most $O(1/ \alpha)$. This directly implies the following
theorem:

\begin{theorem}
  For any graph $G$ whose leverage scores are all $ > \alpha$, graph $G$
  has arboricity $O(1/ \alpha)$.
\end{theorem}

Since many sparsifiers are based on leverage score sampling, we can
immediately apply this result to bound the arboricity of these
sparsifiers. This is codified in the following corollaries:

\begin{corollary} 
  Any $\eps$-sparsifier constructed using the procedure of Spielman and
  Srivastava~\cite{SpielmanS08} has arboricity at most $O(\log n /
  \eps^2$).
\end{corollary}

Srisvastava sparsifiers are 

We do this by using the following core theorem:

\begin{theorem} Let 

Spielman and Srivastava proved in \cite{SpielmanS08} that every graph
has a $(1+\eps)$ spectral sparsifier with $O\left(\frac{n \log n}{\epsilon^2}\right)$ edges using an edge sampling routine. Batson, Spielman, and Srivastava proved in \cite{BatsonSS09} that there exist $(1+\epsilon)$-spectral sparsifiers of graphs with $O(n/\epsilon^2)$ edges.  Furthermore, the Marcus-Spielman-Srivastava proof of the Kadison-Singer conjecture in \cite{MarcusSS13} can be used to show that the edge sampling routine of \cite{SpielmanS08} gives an $(1+\epsilon)$-spectral sparsifier with $O(n/\epsilon^2)$ edges, with non-zero probability. \cite{Srivastava13}

\tim{Insert definition and comments about Leverage Score.}

Our core result is showing that for any graph, the set of edges with
leverage score greater than $\alpha$ has arboricity at least
$\frac{1}{\alpha}$. Since sparsifiers. 
This has the following immediate corollaries:

\begin{corollary} Any Spielman-Srivastava sparsifier has arboricity at
  most $O\left(\frac{\log n}{\epsilon^2}\right)$.
\end{corollary}

\begin{corollary}
The Marcus-Spielman-Srivastava sparsifiers has arboricity at most $O(1/\epsilon^2)$
\end{corollary}
\begin{corollary}
  The Kelner-Levin semi-streaming sparsifiers has at most
  $O(log^2 n /\epsilon^2)$ arboricity.
\end{corollary}
\begin{corollary}
  The Chu-Gao-Peng-Sachdeva-Sawlani-Wang resistance sparsifiers has at most
  $O(n^{o(1)}/\epsilon)$ arboricity.
\end{corollary}
This is as tight a bound as we can hope to get for each class of
sparsifiers.

Our result can be applied to approximating cut queries efficiently.
As shown by Andoni, Krauthgamer, and Woodruff in \cite{AndoniKW14}, any sketch of a graph that w.h.p. preserves all cuts in an $n$-vertex graph must be of size $\Omega(n/\epsilon^2)$ bits.
We show that the Spielman-Srivastava sparsifiers, in addition to achieving nearly optimal construction time and storage space, can also be made to achieve the nearly optimal query time $O\left(|S|\frac{\log n}{\epsilon^2}\right)$ when estimating the boundary of $S \subseteq V$, compared to the trivial query time of $O\left(n\frac{\log n}{\epsilon^2}\right)$.
