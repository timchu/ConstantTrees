\section{Low Arboricity Spectral Sparsifiers}\label{sec:main}

In this section, we prove Theorem~\ref{thm:main} and its associated
corollaries. First we establish some preliminary lemmas. Throughout, we assume graph
$G$ is connected, and $G = (V, E, c)$ where $c_e \forall e \in E$
represents the edge weights (conductances) of graph $G$. 
\begin{lemma} \label{foster}
\textit{(Foster's Resistance Theorem)} Let $G = (V, E, c)$ be any graph. Then 
\begin{equation}
\sum_{e \in E} R_ec_e = n-1.
\end{equation}
where $n:=|V|$. \text{\cite{Foster61}}
\end{lemma}
This is equivalent to saying: the sum of leverage scores in any graph is
exactly $n-1$, which is the rank of $L_G$.

\begin{lemma} \label{lem:monotone}
\textit{(Subgraph Effective Resistances are Higher than Graph Effective
    Resistances)} Let $H$ be a subgraph of $G = (V, E, c)$, where $L_H$ is treated as a linear operator from $\mathbb{R}^{|V|}$ to $\mathbb{R}^{|V|}$. Then

\begin{equation}\label{lem} 
x^T L_H^+ x \geq x^T L_G^+ x
\end{equation}

for all $x \in \mathbb{R}^{|V|}$ orthogonal to the nullspace of $L_H$. 

\end{lemma}

Lemma~\ref{lem:monotone} is known in the literature.
%but we provide a proof
% here for completeness.
% \begin{proof}
% Let $x = L_G^{\frac{1}{2}}y$ and $y=(L_G^+)^\frac{1}{2}x$. Now Equation \ref{lem} is equivalent to the equation
% 
% \begin{equation}
% y^T \left(L_G^\frac{1}{2} L_H^+ L_G^\frac{1}{2}\right) y \geq y^T y
% \end{equation}
% 
% holding true for all vectors $y \in \mathbb{R}^{|V|}$.
% 
% Since $x$ is assumed to be orthogonal to the nullspace of $L_H$ (which equals the nullspace of $L_H^+$), it follows that $y$ is orthogonal to the nullspace of $L_G^\frac{1}{2} L_H^+ L_G^\frac{1}{2}$. Therefore it suffices to show that all the non-zero eigenvalues of $L_G^\frac{1}{2} L_H^+ L_G^\frac{1}{2}$ are greater than $1$. 
% This is equivalent to showing that all the non-zero eigenvalues of $\left(L_G^+ \right)^\frac{1}{2} L_H \left(L_G^+ \right)^\frac{1}{2}$ are less than $1$, a fact which follows immediately from Rayleigh monotonicity.
% \end{proof}
Now we are ready to prove Lemma~\ref{lem:main}, which states that the subgraph
$H$ of graph $G$ has arboricity $O(1/\alpha)$, where $H$ consists of the edges with leverage score in
$G$ greater than $\alpha$.
\begin{proof} (of Lemma~\ref{lem:main})
  We show that $H$ is uniformly sparse.  By Lemma~\ref{lem:lowarb}, this would prove the
  arboricity is $O(1/ \alpha)$.
  Let $S$ be any vertex subset of $V(G)$. Then: we know that the sum of
  $c_e R_e(G)$ on $S$ is at most $|S|-1$, by Foster's Theorem and
  Lemma~\ref{lem:monotone}.

  Therefore,  the set of edges with $c_e R_e(G) > \alpha$ must have at
  most $O(|S|/\alpha)$ edges between two vertices in $S$, by
  Markov's inequality.  This holds for all subsets $S$, and thus implies
  uniform sparsity of $O(1/\alpha)$
\end{proof}
Theorem~\ref{thm:main} follows directly from Lemma~\ref{lem:main}.

\begin{theorem}
Marcus-Spielman-Srivastava sparsifiers are $O(1/\epsilon^2)$-uniformly sparse.
\end{theorem}

\begin{proof} 
For each edge included in the graph by the Marcus-Spielman-Srivastava
sampling scheme, they're designed with the property that each edge is
sampled with probability proportional to their leverage score, and each
edge in the final graph has leverage score at least
$O(\eps^2)$~\cite{Srivastava13, MarcusSS13}.

Therefore, by Theorem~\ref{thm:main}, they have a arboricity of at most $O(1/\eps^2)$.
\end{proof}
This proves Corollary~\ref{cor:mss}. Now we prove that Spielman
Srivastava sparsifiers are uniformly sparse.

\begin{theorem} 
Spielman-Srivastava sparsifiers have arboricity
$O\left(\frac{\log n}{\epsilon^2}\right)$.
\end{theorem}
\begin{proof}
Spielman Srivastava sparsifiers are designed such that the leverage
score of any edge in the final graph is at most
$O(\eps^2 / \log n)$~\cite{SpielmanS08}.
This implies that they have arboricity
at most $O(1/\eps^2)$.
\end{proof}
This proves Corollary~\ref{cor:ss}.

\begin{theorem} 
Kelner-Levin spectral sparsifiers, in the semi-streaming setting, have arboricity $O( \log n /
    \eps^2)$.
\end{theorem}
\begin{proof}
This follows on Theorem~\ref{thm:main} and the bounds on the leverage score of edges in the
Kelner-Levin sparsifier established in Section~\ref{sec:kl}.
\end{proof}

\begin{theorem} 
Chu-Gao-Peng-Sachdeva-Sawlani-Wang resistance sparsifiers can be written
as the union of a graph with nearly linear number of edges, and a graph
with arboricity bounded by
$O(\log^{O(1)}n / \eps)$.
\end{theorem}
\begin{proof}
This follows from Theorem~\ref{thm:main}, and the remarks in
Chu-Gao-Peng-Sachdeva-Sawlani-Wang sparsifier established in
Section~\ref{sec:res-sparse}.
\end{proof}
