\section{The Main Result}

First we establish some preliminary lemmas.
\begin{lemma} \label{foster}
\textit{(Foster's Resistance Theorem)} Let $G = (V, E, c)$ be any graph. Then 
\begin{equation}
\sum_{e \in E} R_ec_e = n-1.
\end{equation}
where $n:=|V|$. \text{\cite{Foster61}}
\end{lemma}

\begin{lemma} \label{monotone}
\textit{(Effective Resistances of edges in a subgraph are higher than in the original graph)} Let $H$ be a subgraph of $G = (V, E, c)$, where $L_H$ is treated as a linear operator from $\mathbb{R}^{|V|}$ to $\mathbb{R}^{|V|}$. Then

\begin{equation}\label{lem} 
x^T L_H^+ x \geq x^T L_G^+ x
\end{equation}

for all $x \in \mathbb{R}^{|V|}$ orthogonal to the nullspace of $L_H$. 

\end{lemma}

\begin{proof}
Let $x = L_G^{\frac{1}{2}}y$ and $y=(L_G^+)^\frac{1}{2}x$. Now Equation \ref{lem} is equivalent to the equation

\begin{equation}
y^T \left(L_G^\frac{1}{2} L_H^+ L_G^\frac{1}{2}\right) y \geq y^T y
\end{equation}

holding true for all vectors $y \in \mathbb{R}^{|V|}$.

Since $x$ is assumed to be orthogonal to the nullspace of $L_H$ (which equals the nullspace of $L_H^+$), it follows that $y$ is orthogonal to the nullspace of $L_G^\frac{1}{2} L_H^+ L_G^\frac{1}{2}$. Therefore it suffices to show that all the non-zero eigenvalues of $L_G^\frac{1}{2} L_H^+ L_G^\frac{1}{2}$ are greater than $1$. 
This is equivalent to showing that all the non-zero eigenvalues of $\left(L_G^+ \right)^\frac{1}{2} L_H \left(L_G^+ \right)^\frac{1}{2}$ are less than $1$, a fact which follows immediately from Rayleigh monotonicity.
\end{proof}

\begin{lemma} \label{sumresist}
Let $V' \subset V$. Let $G'$ be the subgraph of $G$ induced by $V'$, and let $E'$ be the edges of the induced subgraph. 
Then 
\begin{equation}
\sum_{e \in E'} R_e(G)c_e \leq |V'|-1
\end{equation}
\end{lemma}

\begin{proof} Note that $R_e = \frac{1}{2} \chiv L_G \chiv$, where $\chiv$ is defined as in Equation \ref{chiv}. Since $e$ is an edge of subgraph $H$, it follows that $\chiv$ is orthogonal to the nullspace of $L_H$. Thus we can apply Lemma \ref{monotone} and Lemma \ref{foster} to show that

\begin{equation}
\sum_{e \in E'} R_e(G)c_e \leq \sum_{e \in E'} R_e(G')c_e =|V'|-1
\end{equation}

as desired.
\end{proof}

\begin{theorem}\label{main}
Marcus-Spielman-Srivastava sparsifiers are $O(1/\epsilon^2)$-uniformly sparse.
\end{theorem}

\begin{proof} 
For each edge included in the graph by the Marcus-Spielman-Srivastava sampling scheme, they're included in the graph with weight $\frac{c_e\epsilon^2}{p_e(Cn)}$.  Therefore, by Lemma \ref{effectresist}, the value of $R_e(H)c_e(H)$ on edge $e$ is within a $(1+\epsilon)$ multiple of

\begin{equation}
R_e(G)c_e(H) = R_e \cdot \frac{c_e \epsilon^2}{(Cn)p_e} = R_e \cdot \frac{c_e\epsilon^2}{Cn} \cdot\frac{(n-1)}{R_ec_e} \geq \epsilon^2/(2C).
\end{equation}

Here, $R_e(H)$ and $c_e(H)$ denote the effective resistance and conductance of edge $e$ in graph $H$ respectively, and $R_e$ and $c_e$ denote the effective resistance and conducatnce of edge $e$ in graph $G$ respectively.

\vspace{2 mm}
Using Lemma \ref{sumresist}, it follows that the subgraph induced by $V'$ has no more than $2C(|V'|-1)/\epsilon^2$ edges. This implies that any subgraph of a Marcus-Spielman-Srivastava sparsifier is sparse.
\end{proof}
\begin{corollary}
Marcus-Spielman-Srivastava sparsifiers have $O(1/\epsilon^2)$ arboricity.
\end{corollary}

\begin{theorem} 
Spielman-Srivastava sparsifiers are $O\left(\frac{\log n}{\epsilon^2}\right)$-uniformly sparse.
\end{theorem}
\begin{proof}
The proof is identical to the the proof of Theorem \ref{main}, with $Cn$ replaced with $Cn\log n$.
\end{proof}
\begin{corollary}
Spielman-Srivastava sparsifiers have arboricity $O(\frac{\log n}{\epsilon^2})$.
\end{corollary}

