\begin{abstract}
We show that every graph has a spectral sparsifier with a
constant arboricity. Moreover, we show that a range of $\eps$-spectral-sparsifiers have nearly constant arboricity
for constant $\epsilon$.
This includes the spectral sparsifiers of Spielman and Srivastava, the semi-streaming
spectral sparsifiers of Kelner-Levin and Kyng-Pachocki-Peng-Sachdeva,
 and more.
A wide range of algorithms that are otherwise time-consuming, are known
to be efficient on graphs of low arboricity, and therefore apply to any
low-arboricity sparsifier. Examples of such algorithms include approximate
vertex cover, approximate dominating set, approximate cut-queries, and more.
Researchers have already used spectral sparsifiers with low
arboricity to find fast algorithms for
approximate undirected bipartite min-cut.  
In this paper, we also prove that the low arboricity of the Spielman
Srivastava spectral sparsifier lets us compute any cut in a graph
in time proportional to the vertex set size of the smaller side
of the cut, with nearly linear pre-processing time.

We hope that the
efficiency of certain algorithms on graphs of low arboricity will
continue to see algorithmic use.
Our proofs and results hinge on one key lemma, with a simple proof.
It states: for any graph $G$, the subgraph of edges with leverage score
$ > \alpha$ has arboricity $< O(1/ \alpha)$. This is sufficient to imply
the low arboricity of a wide range of spectral sparsifiers.

\end{abstract}
