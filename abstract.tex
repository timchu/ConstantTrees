\begin{abstract}
We show that every graph has a spectral sparsifier with a
constant arboricity. Moreover, we show that a range of $\eps$
sparsifiers have nearly constant arboricity
for constant $\epsilon$.
This includes the spectral sparsifiers of Spielman and Srivastava, the semi-streaming
spectral sparsifiers of Kelner-Levin and Kyng-Pachocki-Peng-Sachdeva,
 and more.
A wide range of algorithms that are otherwise time-consuming, are known
to be efficient on graphs of low arboricity. This includes approximate
vertex cover, approximate dominating set, cut-queries, and more.
Previously, searching for classes of dynamic sparsifiers with low
arboricity have allowed researchers to find fast algorithms for
approximate undirected bipartite min-cut.

Further, we discuss how the low arboricity of the Spielman
Srivastava spectral sparsifier lets us compute any cut in a graph
in time proportional to the vertex set size of the smaller side
of the cut, with nearly linear pre-processing time. We hope that the
efficiency of certain algorithms on graphs of low arboricity will
continue to see algorithmic use.
At the heart of our result is a simple statement, with a simple proof.
It states: for any graph $G$, the subgraph of edges with leverage score
$ > \alpha$ has arboricity $< O(1/ \alpha)$. This is sufficient to imply
the low arboricity of all the sparsifiers mentioned above.

\end{abstract}
