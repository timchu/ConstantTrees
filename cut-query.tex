\section{Applications to Approximating Cut Queries}

\begin{definition}
	We say that a total ordering $\dot{<}$ of the vertices of a graph is \emph{$c$-treelike} if every vertex $u$ has at most $c$ neighbors $v$ such that $u~\dot{<}~v$. 
\end{definition}
\begin{lemma}
	\label{lem_treelike}
	Every graph with arboricity $c$ has a $2c$-treelike ordering.
  Moreover, this ordering can be computed in nearly linear time.
\end{lemma}
\begin{proof}
	Let $G$ be a graph with arboricity $c$.
	Let $v$ be the minimum degree vertex of $G$.
	Note that $d(v) \leq 2c$.
	We set $v$ to be the smallest in the ordering $\dot{<}$ and then
  recursively construct the remainder of the ordering on $G' = G \
  \{v\}$. The ordering can be computed in $O(n \log n + m)$ time by
  using a Fibonacci heap priority queue, and updating degrees as we
  remove the minimum degree vertex $v$.
\end{proof} 
\begin{lemma} \label{lem:cut}
	Let $G = (V, E)$ have arboricity $c$.
	After preprocessing in linear time and space, we can answer queries about the boundaries of subsets of vertices of $G$ in $O(ck)$ time, where $k$ is the size of the queried subset.
\end{lemma}
\begin{proof}
	Using Lemma \ref{lem_treelike} we first compute a $2c$-treelike ordering $\dot{<}$ of $V$.
	For every vertex $u \in V$, we store a list of edges $(u, v) \in E$ such that $u~\dot{<}~v$.
	We also compute and store the weighted degree $wd(v)$ for every vertex of $G$.
	
	Assume we are given $S \subseteq V$.
	We first compute the total weight $s_{internal}$ of edges internal to $S$.
	To this end, for every vertex $u \in S$ we go through its neighbors that are larger in the ordering $\dot{<}$ and sum up the weights of edges that lead to $S$.
	Note that every edge in $S \times S$ will be encountered exactly once.
	The boundary of $S$ can be computed as
	\begin{align*}
		s_{cut} := \left(\sum_{v \in S} wd(v)\right) - 2s_{internal}. 
	\end{align*}
\end{proof}
\begin{theorem}\label{thm:cut-query}
	Given a graph with $n$ vertices, there exists a data structure that:
	\begin{itemize}
	  \item achieves the construction time, storage space, and cut approximation guarantees of Spielman-Srivastava sparsifiers, and
	  \item can compute approximate weights of cuts in $O(k \frac{\log n}{\epsilon^2})$ time, where $k$ is the size of the smaller side of the cut.
	\end{itemize}
\end{theorem}

This is a simple consequence of Lemma~\ref{lem:cut}, and our previous proof that Spielman Srivastava
sparsifiers have $O(\log n / \eps^2)$ arboricity.
